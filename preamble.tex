%%%%%%%%%%%%%%%%%%%%%%%%%%%%%%%%%%%%%%%%%%%%
% pre4.tex -- LaTeX Preamble               %
%                                          %
% Author: Mathias Dannesbo <neic@neic.dk>  %
% Time-stamp: <2016-01-12 18:43:35 (neic)> %
%                                          %
% LuaLaTeX preamble with modern packages.  %
%                                          %
%%%%%%%%%%%%%%%%%%%%%%%%%%%%%%%%%%%%%%%%%%%%

% The special requirements for this to work is:
% - the serif fontface Arno Pro from Adobe,
% - the sans serif fontface Gill Sans Nova,
% - the math fontface Cambria Math from Microsoft and
% - python with
% - Pygments.
% The TeX-engine should also be run with -shell-escape.

%\includeonly{resultsdiscussion}

\RequirePackage[l2tabu, orthodox]{nag}
%\listfiles
\documentclass[a4paper,12pt,twoside,article,table,english,draft]{memoir}[2016/05/16]

% Include etoolbox and luacode for various tools used throughout the preamble
\usepackage{etoolbox, luacode}

%%%%%%%%%%%%%%%%
%%% Geometry %%%
%%%%%%%%%%%%%%%%

% Bringhurst 2.4.8 Never begin a page with the last line of a
% multi-line paragraph. Memoir 3.5 Sloppybottom.
\setlength{\topskip}{1.6\topskip} % memoir
\checkandfixthelayout % memoir
\sloppybottom % memoir

% Bringhurst 4.3.3 [Notes:] Use superscripts in the text but full-size
% numbers in the notes themselves. Memoir 12.1.2 [Footnotes:] Styling.
\setlength{\footmarkwidth}{-1sp} % memoir
\setlength{\footmarksep}{0em} % memoir
\footmarkstyle{#1 } % memoir

% Use the same footmarkstyle with threeparttable
\usepackage[online]{threeparttable}
\appto\TPTnoteSettings{\footnotesize} % threeparttable (etoolbox)


% Setup the use of subfloats in figures and tables. Use
% \subbottom[<caption>]{\includegraphics{…} \label{…}} for a subfigure with a
% caption beneath. Use \subtop[…]{…} for one above. Use \subcaption{<caption>}
% in a minipage.
% Memoir 10.9 Subcaptions
\newsubfloat{figure} % memoir
\newsubfloat{table} % memoir

% Use sans-serif for captions and subcaptions. Make the caption name and counter
% smallcaps.
% Memoir 10.6 Caption styling and 10.9 Subcaptions.
\captionnamefont{\sffamily\scshape} % memoir
\captiontitlefont{\sffamily} % memoir
\subcaptionlabelfont{\sffamily} % memoir
\subcaptionfont{\sffamily} % memoir
% Make the sub* counter letters without any parentheses. Also make it smallcaps
% for \*ref.
\renewcommand{\thesubfigure}{\scshape\alph{subfigure}}
\renewcommand{\thesubtable}{\scshape\alph{subtable}}
% However this also changes the label printed right before the caption to
% smallcaps. To revert this we overwrite the memoir-internals \@thesub*. The
% default is to have \@thesubfigure defined as
% \thesubfigure\if@tightsubcap\hskip\subfloatlabelskip\else\space\fi. We change
% out \thesubfigure (which we just made smallcaps) with \Alph{subfigure} to get
% uppercase letters. This is undocumented in the memoir, but the memoir manual
% states that the subcaptions is copied from the subfigure packages with some
% name changes. Section 5.5 Subfigure Constants of the subfigure manual
% describes \thesubfigure, \@thesubfigure et. al. The default (with few a few
% name changes) is taken from memoir.dtx.
\makeatletter
\def\@thesubfigure{\Alph{subfigure}%
  \if@tightsubcap\hskip\subfloatlabelskip\else\space\fi}
\def\@thesubtable{\Alph{subtable}%
  \if@tightsubcap\hskip\subfloatlabelskip\else\space\fi}
\makeatother

% Use micro-typography.
% Bringhurst 9.4.1 Use the best available justification engine
\usepackage{microtype}
\microtypesetup{final} % microtype


%%%%%%%%%%%%%%%%%%%%%%%%%%%%%%%%%%
%%% Fonts and misc. typography %%%
%%%%%%%%%%%%%%%%%%%%%%%%%%%%%%%%%%

\usepackage{fontspec}

% Create a flag, fancyfonts. If this is set to false, Arno, Gill Sans Nova and
% Cambria Math is not used. This is helpful if you (or your collaborators) don't
% have the fonts installed. If this is set to false, the functionality of the
% preamble is uncertain and untested.
\newtoggle{fancyfonts} % (etoolbox)
% Uncomment the line you want to use.
\toggletrue{fancyfonts} % (etoolbox)
%\togglefalse{fancyfonts} % (etoolbox)

\iftoggle{fancyfonts}{% etoolbox
  % Set Arno Pro as the main font and Gill Sans as the sans font.
  \setmainfont{Arno Pro}[Ligatures=TeX,Numbers=OldStyle] % fontspec
  \setsansfont{Gill Sans Nova}[Scale=MatchUppercase,Numbers=OldStyle] % fontspec

  % Use \lining to get lining numbers in a scope. Use \swash to get
  % swash style in a scope.
  \newcommand{\lining}{\addfontfeature{Numbers=Lining}} % fontspec
  \newfontface\swash[Ligatures=TeX,Numbers=OldStyle,Style=Swash]{Arno Pro Italic} % fontspec
}{
  % If fancyfonts is not set to true. Fail silently on \lining and \swash.
  \newcommand{\lining}{}
  \newcommand{\swash}{}
} % end \iftoggle{fancyfonts}

% Use \abbr with uppercase instead of \textsc with lowercase for abbreviations.
% E.g. \abbr{NATO} instead of \textsc{nato}. This makes the copyable text
% uppercase and also works in headings when hyperref or bookmark generates a pdf
% bookmark tree. Bringhurst 3.2.2 For abbreviations and acronyms in the midst of
% normal text, use spaced small caps.
\newcommand{\abbr}[1]{\texorpdfstring{%
    {\iftoggle{fancyfonts}{%
        \addfontfeature{Letters=UppercaseSmallCaps}#1}{%
        \luaexec{tex.print((string.gsub('#1','\%u+','\\textsc{\\lowercase{\%0}}')))}}%
    }}{#1}} % hyperref, fontspec, (etoolbox, luacode)


%%%%%%%%%%%%%%%%%%%%
%%% Localization %%%
%%%%%%%%%%%%%%%%%%%%

% Bringhurst 2.1.4 Use a single word space between sentences.
% Bringhurst 2.4.1 At hyphenated line-ends, leave at least two
% characters behind and take at least three forward.
% Bringhurst 2.4.5 Hypenate according to the conventions of the
% language.
\usepackage{polyglossia}
\setdefaultlanguage{english} % polyglossia

%%%%%%%%%%%%%%%%%%
%%% Math setup %%%
%%%%%%%%%%%%%%%%%%

\usepackage{amsmath,unicode-math,mathtools}
\unimathsetup{math-style=ISO,bold-style=ISO} % unicode-math

\iftoggle{fancyfonts}{% (etoolbox)
  % Set Arno Pro as the main math font for the glyphs it has. Use Cambria Math as
  % fallback for the rest. \setmathfont with range={…} and
  % (s)script-features={…} are from unicode-math. Script=Latin is directly from
  % fontspec.
  % Script=Latin tells fontspec to use the OpenType script 'Latin' instead
  % of the expected 'Math'. The 'Math' script is not defined in Arno or Gill
  % Sans Nova as they are not math types.
  % (s)script-features={} refer to script unrelated the OpenType script above.
  % It overwrites unicode-maths normal OpenType feature flags for super- and
  % subscript. It defaults to '+ssty', which is also missing from Arno and Gill
  % Sans Nova.
  \setmathfont{Cambria Math}[Scale=MatchLowercase]
  \setmathfont{Arno Pro}[range={up/{num,latin,Latin,greek,Greek},%
    % Below are all the symbols defined in unicode-math which are also present
    % in Arno Pro, have the right metrics and have no similar symbols missing.
    \mathexclam,\mathplus,\pm,\div,\dagger,\ddagger,\minus,\mathoctothorpe,%
    \mathdollar,\mathpercent,\mathampersand,\mathquestion,\mathatsign,%
    \mathsterling,\mathyen,\euro,\increment,\infty,\checkmark,%
    \less,\equal,\greater,\ne,\leq,\geq,\matheth,\ell,\partial,},%
  Script=Latin, script-features={}, sscript-features={}]
  \setmathfont{Arno Pro Italic}[range={it/{latin,Latin,greek,Greek}},%
  Script=Latin, script-features={}, sscript-features={}]
  \setmathfont{Arno Pro Bold}[range = {bfup/{num,latin,Latin,greek,Greek}},%
  Script=Latin, script-features={}, sscript-features={}]
  \setmathfont{Arno Pro Bold Italic}[range={bfit/{latin,Latin,greek,Greek}},%
  Script=Latin, script-features={}, sscript-features={}]

  % Set Gill Sans Nova as the sf math font.
  \setmathfont{Gill Sans Nova Book}[Scale=MatchUppercase,
  range={sfup/{num,latin,Latin}},% missing greek,Greek
  Script=Latin, script-features={}, sscript-features={}]
  \setmathfont{Gill Sans Nova Book Italic}[Scale=MatchUppercase,%
  range={sfit/{latin,Latin}},% missing num,greek,Greek
  Script=Latin, script-features={}, sscript-features={}]
  \setmathfont{Gill Sans Nova Bold}[Scale=MatchUppercase,%
  range={bfsfup/{num,latin,Latin}},% missing greek,Greek
  Script=Latin, script-features={}, sscript-features={}]
  \setmathfont{Gill Sans Nova Bold Italic}[Scale=MatchUppercase,%
  range={bfsfit/{latin,Latin,greek,Greek}},% missing num
  Script=Latin, script-features={}, sscript-features={}]

  % Set Arno Pro with the swash style as the math cal font
  \setmathfont{Arno Pro Italic}[range={cal,bfcal}, RawFeature={+swsh},
  Script=Latin, script-features={}, sscript-features={}]
}{} % end \iftoggle{fancyfonts}


% Use siunitx to typeset numbers and units with \num{}, \si{}, \SI{}{} etc.
\usepackage{siunitx}
\sisetup{%
  % With detect-all, the siunitx macros automatically detects the font around it
  % and changes its output accordingly.
  detect-all,
  % The -micro setup is to appropriate µ when using fontspec and an incompatible
  % font. Arno and Gill Sans Nova works, but the fonts used with
  % fancyfonts=false, does not. See section 7.6 Symbols and XeTeX in
  % siunitx.pdf. \angstrom and \ohm works without any modification.
  text-micro = {µ}, math-micro=\text{µ},
} % siunitx

%%%%%%%%%%%%%%%%%%%%%%%%%%%
%%% Colors and graphics %%%
%%%%%%%%%%%%%%%%%%%%%%%%%%%
\usepackage{xxcolor} % xxcolor is an extension for xcolor which is an extension
                     % for color. xxcolor is shipped with pgf.

% The Tableau 20 color palette is a nice list of colors for data visualization.
\definecolor{tblue}{RGB}{31,119,180}    \definecolor{tlightblue}{RGB}{174,199,232}
\definecolor{torange}{RGB}{255,127,14}  \definecolor{tlightorange}{RGB}{255,187,120}
\definecolor{tgreen}{RGB}{44,160,44}    \definecolor{tlightgreen}{RGB}{152,223,138}
\definecolor{tred}{RGB}{214,39,40}      \definecolor{tlightred}{RGB}{255,152,150}
\definecolor{tpurple}{RGB}{148,103,189} \definecolor{tlightpurple}{RGB}{197,176,213}
\definecolor{tbrown}{RGB}{140,86,75}    \definecolor{tlightbrown}{RGB}{196,156,148}
\definecolor{tpink}{RGB}{227,119,194}   \definecolor{tlightpink}{RGB}{247,182,210}
\definecolor{tgray}{RGB}{127,127,127}   \definecolor{tlightgray}{RGB}{199,199,199}
\definecolor{tolive}{RGB}{188,189,34}   \definecolor{tlightolive}{RGB}{219,219,141}
\definecolor{tteal}{RGB}{23,190,207}    \definecolor{tlightteal}{RGB}{158,218,229}
% all (xx)color

% Use tikz for drawing and pgfplots for plotting.
\usepackage{tikz,pgfplots}

% Freeze the pgfplots features
\pgfplotsset{compat=1.15} % pgfplots

% Save the tikz and pgfplots figures as external files for faster compilation.
\usetikzlibrary{external} % tikz
\tikzexternalize[prefix=tikz/] % tikz

% Create a cyclelist of the first ten Tableau 20 colors for pgfplots.
\pgfplotscreateplotcyclelist{tcolor}{%
  {solid, color=tblue}, {solid, color=torange}, {solid, color=tgreen},
  {solid, color=tred}, {solid, color=tpurple}, {solid, color=tbrown},
  {solid, color=tpink}, {solid, color=tgray}, {solid, color=tolive},
  {solid, color=tteal}%
} % pgfplots

% And set that cyclelist as a global standard
\pgfplotsset{cycle list name={tcolor}} % pgfplots

% Transfer siunitx setting to pgf for identical number representation in text
% and figures. See section 7.10 'Transferring settings to pgf' of the siunitx
% manual.
\SendSettingsToPgf % siunitx


%%%%%%%%%%%%%%%%%%%%
%%% TÅGEKAMMERET %%%
%%%%%%%%%%%%%%%%%%%%

%\usepackage{tket}
%\TKsetup{C = {\kern-0.1ex\scalebox{0.94}{\raisebox{-0.08ex}{\ensuremath{ℂ}}}}} % med Arno Pro og TeX Gyre Pagella math
%\newfontface\bbface[Scale=0.87]{TeX Gyre Termes Math} \TKsetup{C = {\bbface\kern-0.1exℂ}} % fontspec,tket
%\TKsetup{dollar={\$}} % tket
% lining KA$$


%%%%%%%%%%%%%%%%%
%%% Titlepage %%%
%%%%%%%%%%%%%%%%%
% TODO
\setlength{\droptitle}{-3em}
\pretitle{\LARGE\par} \posttitle{\vskip 0.5em}
\newcommand{\supertitle}[1]{\gdef\suP{#1}}
\renewcommand{\maketitlehooka}{\ifx\suP\undefined\begin{center}\else\begin{center} {\scshape\suP}\fi}
    \newcommand{\subtitle}[1]{\gdef\suB{#1}}
    \renewcommand{\maketitlehookb}{\ifx\suB\undefined \end{center}\else\par {\large\scshape\suB}\par\end{center}\fi}

%%%%%%%%%%%%%%
%%% Header %%%
%%%%%%%%%%%%%%
% TODO
\newcommand{\stunum}[1]{\gdef\stuN{#1}}
\copypagestyle{articlehead}{plain}
\makeoddhead{articlehead}{\color{gray}\suB}{}{\color{gray}\thedate}
\pagestyle{articlehead}

%%%%%%%%%%%%%%
%%% resten %%%
%%%%%%%%%%%%%%
% TODO

% Include mhchem to typeset chemistry easily.
\usepackage[version=4]{mhchem}


\usepackage{pdfpages,xspace}
% \usepackage{minted} % requires minted > 2.0-alpha2
% \usemintedstyle{tango}

\usepackage{ragged2e}
\usepackage{csquotes}

% TODO
\DeclarePairedDelimiterX\Set[2]{\lbrace}{\rbrace}{#1\,\delimsize\vert\, #2}

%%%%%%%%%%%%
%%% Help %%%
%%%%%%%%%%%%
% TODO
\usepackage{lipsum}
\usepackage[margin,draft]{fixme}
\fxusetheme{color} % fixme

%%%%%%%%%%%%%%%%%%%%
%%% Bibliography %%%
%%%%%%%%%%%%%%%%%%%%
\usepackage[
  backend=biber,
  sortlocale=auto,
  sortcites=true,
  hyperref=true,
  maxbibnames=99,
  style=numeric,
  sorting=none,
]{biblatex}
\addbibresource{literatur.bib} % biblatex

%%%%%%%%%%%%%%%%%%
%%% References %%%
%%%%%%%%%%%%%%%%%%

% Use \vref, \Vref or some other macro from varioref for references.

% varioref, hyperref, cleverref and bookmarks need to be loaded in this order.
% See cleverref doc. section 13.
\usepackage{varioref}
\usepackage[unicode=true,
    pdfusetitle,
    pdfkeywords={},
    pdfdisplaydoctitle=true,
    ]{hyperref}
\usepackage{cleveref}


%%%%%%%%%%%%%%%%%%%%%%%%
%%% Table of Content %%%
%%%%%%%%%%%%%%%%%%%%%%%%

% Generate a pdf outline with bookmarks.
\usepackage{bookmark}
\usepackage[hyphenation, lastparline, nosingleletter,homeoarchy,rivers,draft]{impnattypo}

\maxtocdepth{section} % memoir


%%%%%%%%%%%%%%%%%%%%%%
%%% Custom changes %%%
%%%%%%%%%%%%%%%%%%%%%%

% Add you custom changes here




\begin{document}
%%%%%%%%%%%%%%%%%%%
%%% Frontmatter %%%
%%%%%%%%%%%%%%%%%%%
\frontmatter
%\thispagestyle{empty}
\supertitle{A super title}
\title{A title}
\subtitle{A sub title}
\author{An-Author}
\stunum{12345678}
\date{\today}
\maketitle

\clearpage
%\include{acknowledgements}
%\include{abstract}
%\include{abbreviations}

\tableofcontents

%%%%%%%%%%%%%%%%%%
%%% Mainmatter %%%
%%%%%%%%%%%%%%%%%%
\mainmatter
\chapter{Copy this \abbr{NATO}}
\section {Copy this \textsc{NATO}}
The \textsc{nato} office fiddo 1234 æøåÅ.\\
The \abbr{NATO} office fiddo 1234 æøåÅ.\\
The {\addfontfeature{Letters=UppercaseSmallCaps}NATO} office fiddo 1234 æøåÅ.\\
{\addfontfeature{Ligatures=TeX}The {\addfontfeature{Letters=UppercaseSmallCaps}NATO} office fiddo 1234 æøåÅ.}\\


\chapter{Line lenght test}
abcdefghijklmnopqrstuvwxyz abcdefghijklmnopqrstuvwxyz abcdefghijklmnopqrstuvwxyz
abcdefghijklmnopqrstuvwxyz

\noindent a\-b\-c\-d\-e\-f\-g\-h\-i\-j\-k\-l\-m\-n\-o\-p\-q\-r\-s\-t\-u\-v\-w\-x\-y\-z\-a\-b\-c\-d\-e\-f\-g\-h\-i\-j\-k\-l\-m\-n\-o\-p\-q\-r\-s\-t\-u\-v\-w\-x\-y\-z\-a\-b\-c\-d\-e\-f\-g\-h\-i\-j\-k\-l\-m\-n\-o\-p\-q\-r\-s\-t\-u\-v\-w\-x\-y\-z\-a\-b\-c\-d\-e\-f\-g\-h\-i\-j\-k\-l\-m\-n\-o\-p\-q\-r\-s\-t\-u\-v\-w\-x\-y\-z
\chapter{Font test}
\foreach \x in {1234567890,abcdefghijklmnopqrstuvwxyz,αβξδεφγηικλμνοπθρστυςωχψζ,+−-=≈.≥≤<>}
{
  \section{With alfabet: \x}
  \newcommand{\y}{\x\MakeTextUppercase{\x}}
  \small
  \noindent\y\space{\tiny normal}\newline
  $\mathup{\y}$\space{\tiny math up}\newline
  \textit{\y}\space{\tiny textit}\newline
  $\y$\space{\tiny math}\newline
  \textbf{\y}\space{\tiny textbf}\newline
  $\mathbfup{\y}$\space{\tiny mathbfup}\newline
  \textbf{\textit{\y}}\space{\tiny textbfit}\newline
  $\mathbfit{\y}$\space{\tiny mathbfit}\newline
  \textsf{\y}\space{\tiny textsf}\newline
  $\mathsfup{\y}$\space{\tiny mathsfup}\newline
  \textsf{\textit{\y}}\space{\tiny textsfit}\newline
  $\mathsfit{\y}$\space{\tiny mathsfit}\newline
  \textbf{\textsf{\textit{\y}}}\space{\tiny textbfsfit}\newline
  $\mathbfsfit{\y}$\space{\tiny mathbfsfit}\newline
  {\swash\y}\space{\tiny swash}\newline
  $\mathcal{\y}$\space{\tiny mathcal}\newline
  \newpage
}

\chapter{Number test}
{
  \newcommand{\y}{1234567890}
  \small
  Should all be oldstyle upright:\newline
  \noindent\y\space{\tiny normal}\newline
  Should all be lining upright:\newline
  {\lining\y}\space{\tiny normal lining}\newline
  $\y$\space{\tiny math}\newline
  $\mathup{\y}$\space{\tiny mathup}\newline
  Should all be oldstyle italic:\newline
  \textit{\y}\space{\tiny textit}\newline
  Should all be lining italic:\newline
  {\lining\textit{\y}}\space{\tiny textit lining}\newline
  $\mathit{\y}$\space{\tiny mathit}\newline
  Should all be oldstyle bold:\newline
  \textbf{\y}\space{\tiny textbf}\newline
  Should all be lining bold:\newline
  {\lining \textbf{\y}}\space{\tiny textbf lining}\newline
  $\mathbf{\y}$\space{\tiny mathbf}\newline
  $\mathbfup{\y}$\space{\tiny mathbfup}\newline
  Should all be oldstyle bold italic:\newline
  \textbf{\textit{\y}}\space{\tiny textbfit}\newline
  Should all be lining bold italic:\newline
  {\lining \textbf{\textit{\y}}} \space{\tiny textbfit lining}\newline
  $\mathbfit{\y}$\space{\tiny mathbfit}\newline
  Should all be sans oldstyle:\newline
  \textsf{\y}\space{\tiny textsf}\newline
  Should all be sans lining:\newline
  \textsf{\lining\y}\space{\tiny textsf lining}\newline
  $\mathsf{\y}$\space{\tiny mathsf}\newline
  $\mathsfup{\y}$\space{\tiny mathsfup}\newline
  Should all be sans oldstyle italic:\newline
  \textsf{\textit{\y}}\space{\tiny textsfit}\newline
  Should all be sans lining italic:\newline
  \textsf{\lining\textit{\y}}\space{\tiny textsfit lining}\newline
  $\mathsfit{\y}$\space{\tiny mathsfit}\newline
  Should all be sans oldstyle bold:\newline
  \textbf{\textsf{\y}}\space{\tiny textbfsf}\newline
  Should all be sans lining bold:\newline
  \textbf{\textsf{\lining\y}} \space{\tiny textbfsf lining}\newline
  $\mathbfsf{\y}$\space{\tiny mathbfsf}\newline
  Should all be sans oldstyle bold italic:\newline
  \textbf{\textsf{\textit{\y}}}\space{\tiny textbfsfit}\newline
  Should all be sans lining bold italic:\newline
  \textbf{\textsf{\lining\textit{\y}}} \space{\tiny textbfsfit lining}\newline
  $\mathbfsfit{\y}$\space{\tiny mathbfsfit}\newline
}

\tightlists
\begin{itemize}
\item Ligning
\item Math up
\item lining, textit
\item math
\end{itemize}

{\lining A5} $\textup{A5}$ {\lining \textit{G4}} $G4$

\begin{itemize}
\item mhchem
\item math
\end{itemize}

\ce{Mg^{2+}}$Mg^{2+}$

\begin{itemize}
\item textbf
\item mathbf
\end{itemize}

\textbf{c8}$\mathbf{c8}$


\chapter{Math symbol test}

\begin{align}
  \mathexclam\mathcomma\mathcolon\mathsemicolon \dblcolon \coloneqq \Coloneqq \eqqcolon\\
  \hat x \tilde x\\
  \prod\coprod\sum\int\iint\\
  \mathplus\pm\mp\cdotp\cdot x\times x\div\dagger\ddagger\smblkcircle %
  \fracslash\minus\divslash\vysmblkcircle\triangleright\triangleleft\\
  \mathoctothorpe\mathdollar\mathpercent\mathampersand\mathperiod%
  \mathslash\mathquestion\mathatsign\backslash\mathsterling\mathyen%
  \neg\upoldkoppa\enleadertwodots\unicodeellipsis\prime\dprime\euro%
  \increment\infty\mdlgblksquare\blacktriangleright\blacktriangleleft%
  \mdlgblkdiamond\fisheye\mdlgwhtlozenge\checkmark\\
  \nless\less\equal\doteq\greater\leftarrow\rightarrow\approx\simeq\ne\leqq\leq\geq\gg\\
  \matheth\ell\partial
\end{align}

\chapter{Math test}
\begin{align*}
  A &= \sum_{a=100}^{k} k \cdot 21\\
  B &= \bigg( \frac{1}{2} \bigg)\\
  C &= \left( \frac{1}{\frac{5}{asdfg}} \right)\\
  D &= \bigg[ \frac{1}{2} \bigg]\\
  E &= \Set{a}{200 ∈ ℤ}\\
  F &= a^{b} + a^{b^{c}} a^{b^{c^{d}}}\\
  G &= \Bigg( \bigg( \Big( \big( (x) \big) \Big) \bigg) \Bigg)\\
  H &= \Biggl( \biggl( \Bigl( \bigl( (x) \bigr) \Bigr) \biggr) \Biggr)\\
  I &= \left( \left( \left( \left( \left( \left( x \right) \right) \right) \right) \right)  \right)
\end{align*}


\chapter{Serif and Sans-Serif text}
Lorem \textsf{ipsum} dolor sit \textsf{amet}, consectetuer adipiscing elit. Ut
purus elit, vestibulum ut, placerat ac, adipiscing vitae, felis. Curabitur
dictum gravida mauris. Nam arcu libero, nonummy eget, consectetuer id, vulputate
a, magna. Donec vehicula augue eu neque. Pellentesque habitant morbi tristique
senectus et netus et malesuada fames ac turpis egestas. Mauris ut leo. Cras
viverra metus rhoncus sem.\textsf{ Nulla et lectus vestibulum urna fringilla
  ultrices.} Phasellus eu tellus sit amet tortor gravida placerat. Integer
sapien est, iaculis in, pretium quis, viverra ac, nunc. Praesent eget sem vel
leo ultrices bibendum. Aenean faucibus. Morbi dolor nulla, malesuada eu,
pulvinar at, mollis ac, nulla. Curabitur auctor semper nulla. Donec varius orci
eget risus. Duis nibh mi, congue eu, accumsan eleifend, sagi is quis, diam. Duis
eget orci sit amet orci dignissim rutrum.

\foreach \x in {a,b,c,d,e,f,g,h,i,j,k,l,m,n,o,p,q,r,s,t,u,v,w,x,y,z}
{\MakeTextUppercase{\x}\textsf{\MakeTextUppercase{\x}}}

\foreach \x in {a,b,c,d,e,f,g,h,i,j,k,l,m,n,o,p,q,r,s,t,u,v,w,x,y,z}
{\x\textsf{\x}}

\chapter{A chapter}
\lipsum[1]
\section{A section}
\lipsum[2-5]
\subsection{A subsection}
\lipsum[6-9]
\subsubsection{A subsubsection}
\lipsum[10]
\paragraph{A paragraph}
\lipsum[11]
\subparagraph{A subparagraph}
\lipsum[12-50]

%\include{introduction}
%\include{materialmethods}
%\include{resultsdiscussion}
%\include{conclusion}

\raggedright
\defbibheading{chapter}[\refname]{\chapter*{#1}\addcontentsline{toc}{chapter}{\refname}}
\printbibliography[heading=chapter]

\clearpage \appendix
%\include{appendix/consetrationcalculations}
%\include{appendix/additionalmeasurements}

%%%%%%%%%%%%%%%%%%
%%% Backmatter %%%
%%%%%%%%%%%%%%%%%%
\clearpage
\backmatter
% \listoftables
% \listoffigures
% \listoflistings

\end{document}

%%% Local Variables:
%%% coding: utf-8
%%% mode: latex
%%% End:
