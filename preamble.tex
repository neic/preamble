\RequirePackage[l2tabu, orthodox]{nag}
\documentclass[a4paper,final,12pt,oneside,article,table]{memoir}

%% Geometry
%\semiisopage[12]
\setlength{\topskip}{1.6\topskip} % for \sloppybuttom
\checkandfixthelayout
\sloppybottom

%% Typography
\usepackage{polyglossia,microtype,hyperref,amsmath,unicode-math,xcolor,natbib,tket}
\definecolor{zen-red}{HTML}{B23333}    \definecolor{zen-orange}{HTML}{E57A33}
\definecolor{zen-yellow}{HTML}{F0DFAF} \definecolor{zen-green}{HTML}{5F7F5F}
\definecolor{zen-cyan}{HTML}{93E0E3}   \definecolor{zen-blue}{HTML}{336CB2}
\setdefaultlanguage{danish} %polyglossia

\hypersetup{colorlinks=true,linkcolor=zen-red,citecolor=zen-green,urlcolor=zen-orange} % hyperref
\microtypesetup{final,verbose=silent} 
\setmainfont[Ligatures=TeX,Numbers=OldStyle]{Arno Pro} % fontspec
%\setmonofont[Scale=MatchLowercase]{DejaVuSansMono} % fontspec

\unimathsetup{math-style=ISO,bold-style=ISO} % unicode-math
\setmathfont[Scale=MatchLowercase]{Cambria Math} % unicode-math

\newfontface\bbface[Scale=0.87]{TeX Gyre Termes Math} \TKsetup{C = {\bbface\kern-0.1exℂ}} % fontspec,tket

%% Titlepage
\setlength{\droptitle}{-3em}
\pretitle{\LARGE\par} \posttitle{\vskip 0.5em}
\newcommand{\supertitle}[1]{\gdef\suP{#1}}
\renewcommand{\maketitlehooka}{\ifx\suP\undefined\begin{center}\else\begin{center} {\scshape\suP}\fi}
\newcommand{\subtitle}[1]{\gdef\suB{#1}}
\renewcommand{\maketitlehookb}{\ifx\suB\undefined \end{center}\else\par {\large\scshape\suB}\par\end{center}\fi}

%% Header
\newcommand{\stunum}[1]{\gdef\stuN{#1}}
\copypagestyle{articlehead}{plain}
\makeoddhead{articlehead}{\color{gray}\theauthor\ifx\stuN\undefined\else\ifx\stuN\empty\else~(\stuN)\fi\fi}{}{}
\pagestyle{articlehead}

%% Grapics
\usepackage{tikz,pgfplots,tikz-timing}
\usetikzlibrary{mindmap,arrows,positioning,shapes}

%% resten
\usepackage{threeparttable,siunitx,pdfpages,algpseudocode,algorithm,tkvc}
\sisetup{per-mode=symbol}
\makeatletter \renewcommand{\ALG@name}{Algoritme}\makeatother
\usepackage[]{minted} % requires minted > 2.0-alpha2
\usemintedstyle{tango}


%% Help
\usepackage{lipsum}
\usepackage[margin,draft]{fixme} \fxusetheme{color}

\begin{document}
\supertitle{Programmerings Sprog}
\title{\VC-papir}
\subtitle{Genaflevering}
\author{Mathias~Dannesbo}
\stunum{201206106}
\date{\today}
\maketitle

\chapter{dADS1}

\makeatletter
\@for\@tempa:=a,b,c,d,e,f,g,h,i,j,k,l,m,n,o,p,q,r,s,t,u,v,w,x,y,z,A,B,C,D,E,F,G,H,I,J,K,L,M,N,O,P,Q,R,S,T,U,V,W,X,Y,z,α,β,γ,δ,ε,ζ,η,θ,ι,κ,λ,μ,ν,χ,ο,π,ρ,σ,τ,υ,φ,ξ,ψ,ω,Α,Β,Γ,Δ,Ε,Ζ,Η,Θ,Ι,Κ,Λ,Μ,Ν,Χ,Ο,Π,Ρ,Σ,Τ,Υ,Φ,Ξ,Ψ,Ω\do{%
\expandafter\textit{\@tempa}$\@tempa$ }
\makeatother


{\HUGE \VC \CERM}
hello \VC tex$m ≥D$TEXT
\section{\VC Opgave 12 - Potensopløftning}
\textit{Betragt følgende algoritme:} og der henvises til Tabel \ref{tab:hstr} eller \url{http://wiply.neic.dk/pic/bryllup}. \TKET\footnote{Her er en fodnote.}
\[\{ p \geq 0 \} r←1, q←p\{I\}, \mathcal{L}(x)\]
\[\{I ∧ q > 0 \} r=r·x, q= q-1\{I\}\]
\[\{I ∧ ¬(q > 0) \} ⇒ r=x^p.\]
\begin{align*}
    \sum_{i=1}^6 y&= y+1\\
    \mathbb{1}&
\end{align*}

running text $\sum_{i=1}^6 y= yx+1$ running text $Σ_{i=1}^6 y= yx+1$ running text
\begin{align*}
  \sum_{i=1}^6 y&= yx+1\\
  Σ_{i=1}^6 y&= yx+1\\
  I &= (r·x^q = x^p) ∧ (q \geq 0)\\
  \mathbf{I} &= \mathbf{(r·x^q = x^p) ∧ (q \geq 0)}\\
  \mathcal{I} &= (rI·x^q = x^p) ∧ (q \geq 0)\\
  \mathbfcal{I} &= (r·x^q = x^p) ∧ (q \geq 0)
\end{align*}\fxwarning{\texttt{\textbackslash sum}, Σ does not look right}
$=<>≤≥←→*()\Bigg(\Bigg)$

\paragraph{Gennemførelse af bevisbyrder}
\textit{Gennemfør bevisbyrderne.}
\subparagraph{Inden iterationen} De nuværende værdier, begyndelsesværdierne, indsættes i $I$ og det første kriterium kan da bevises: $1·x^p = x^p ⇔ x^p = x^p$. Inputbetingelsen er $p \geq 0$ og da $q←p$ må det andet kriterium, $q \geq 0$, da også gælde.
\subparagraph{Efter hver iteration} Efter en iteration er $r'=r·x$ og $q'=q-1$. Det kan indsættes i $I$:
\[ x' · x^{q'} = x·x·x^{q-1} = x·x^q = x·x^p \]
\[ x\prime · x^{q\prime} = x·x·x^{q-1} = x·x^q = x·x^p \]

\[ x' · x^{q'} = x·x·x^{q-1} = x·x^q = x·x^p \]
\[ x^\prime · x^{q\prime} = x·x·x^{q-1} = x·x^q = x·x^p \] 
Før iterationen var $q>0$ og da $q'=q-1$ må $q' \geq 0$ efter en iteration.
\subparagraph{Outputtet}
Det er lige bevist at en en iteration er $q \geq 0$, $q$ ikke kan være negativ på grund af begyndelsesværdierne og da $r·x^q = x^p$ må $r·x^1 = r = x^p$.

Herved er alle byrder bevist.

\paragraph{Termineringsfunktion}
\textit{Angiv en termineringsfunktion.}

Angives $\mu(x,p,r,q) = q$ som termineringsfunktion gælder følgende
\[I ⇒ \mu(x,p,r,q) \geq 0 \text{ og } I ∧ (q>0) ⇒ \mu(x,p,r,q) > \mu(x,p,r,q) \geq 0 \]
da, $q$ ifølge transitionerne falder med 1 for hver udførelse og det er bevist at $q \geq 0$. Det er dermed en gyldig termineringsfunktion.

\paragraph{Konklusion}
\textit{Konkluder, at algoritmen er korekt.}

Da alle bevisbyrder er bevist og der findes en en gyldig termineringsfunktion, er algoritmen korekt.

\chapter{dProgSprog}

\section*{Excercise 1}
\textit{In this mandatory exercise, you are asked to implement a variadic version of \texttt{proper-list-append}.}

The implementations and calls of the procedures:
\inputminted{scheme}{opg1.ss}
\inputminted{java}{CityImpl.java}
\textit{Key questions:}
\textit{What happens if one of the actual parameters of your variadic version of \texttt{proper-list-append} is the empty list?}

\begin{itemize}
\item The other parameter is returned.
\end{itemize}
\textit{What happens if one of the actual parameters of your variadic version of \textit{proper-list-append} is not a proper list?}
\begin{itemize}
\item It fails to return and cast an exception.
\end{itemize}
\textit{What happens if your variadic version of \textit{proper-list-append} is applied to zero actual parameters?}
\begin{itemize}
\item It returns the empty list.
\end{itemize}
\textit{What happens if your variadic version of \textit{proper-list-append} is applied to one actual parameter?} 
\begin{itemize}
\item It returns that parameter.
\end{itemize}

\section*{Excercise 2}
\textit{The goal of this mandatory exercise is to implement the BNF of the following subset of Scheme, and to program a syntax checker that is self-applicable:}

The syntax checker

There are still some missing pieces in the implementation.

\chapter{Arrangementer}

\section{HSTR}
\begin{table}[htbp]
  % \caption[]{}
  \centering
  \begin{threeparttable}
  \begin{tabular}{l*{4}{r@{}r}}
    \toprule
    Type & \multicolumn{2}{c}{10/11}& \multicolumn{2}{c}{11/12}& \multicolumn{2}{c}{12/13}& \multicolumn{2}{c}{13/14}\\
    \midrule
    Personer & \multicolumn{2}{c}{}& \multicolumn{2}{c}{}& \multicolumn{2}{c}{}& \multicolumn{2}{c}{54}\\[1.5ex]
    Alm. øl                & \tabkas{13}{0} & \tabkas{15}{0}& &---             & &---\\
    Top                    & &---           & &---          & \tabkas{3,43}{0} & \tabkas{6}{0}\\
    Tuborg                 & &---           & &---          & \tabkas{2,53}{0} & \tabkas{4}{0}\\
    Hof                    & &---           & &---          & \tabkas{3,367}{0}& \tabkas{2}{0}\\
    Høker                  & &---           & &---          & \tabkas{1,7}{0}  & &---\\
    Odense                 & &---           & &---          & \tabkas{0,97}{0} & &---\\
    Andre alm. øl\tnote{1} & &---           & &---          & &---             & \tabkas{0}{12}\\[1.5ex]
    Guldøl                 & \tabkas{2}{0}  & \tabkas{2}{0} & &---             & &---\\
    LFP                    & &---           & &---          & \tabkas{0,6}{0}  & \tabkas{0}{6}\\
    Elephant               & &---           & &---          & \tabkas{1}{0}    & &---\\
    Gyldne Dame            & &---           & &---          & \tabkas{0,37}{0} & \tabkas{0}{7}\\
    Blå Thor               & &---           & &---          & \tabkas{0}{0}    & \tabkas{0}{0}\\[1.5ex]
    Vand                   & \tabkas{3}{0}  & \tabkas{3}{0} & &---             & &---\\
    CocaCola               & &---           & &---          & \tabkas{0,4}{0}  & \tabkas{0}{27}\\
    CocaCola Zero          & &---           & &---          & &---             & \tabkas{0}{0}\\
    CocaCola Light         & &---           & &---          & \tabkas{0,13}{0} & &---\\
    Squash                 & &---           & &---          & \tabkas{0,07}{0} & \tabkas{0}{6}\\
    Dansk Vand             & &---           & &---          & \tabkas{0,43}{0} & &---\\
    12xVand                & &---           & &---          & \tabkas{0,08}{0} & \tabkas{0}{6}\\
    \bottomrule
  \end{tabular}
  \begin{tablenotes}
  \item [1] Der blev medbragt en blandet kasse med Høker, Odense, Tuborg Classic og Carls Special.
  \end{tablenotes}
  \end{threeparttable}
  % \legend{}
  \label{tab:hstr}
\end{table}


%\clearpage
\listoftables
\listoffigures
\listoflistings
%\nocite{*}
%\bibliographystyle{dlfltxbbibtex} \Bibliography{bib}
%\clearpage \appendix

\end{document}

%%% Local Variables: 
%%% coding: utf-8
%%% mode: latex
%%% TeX-engine: xetex
%%% End: